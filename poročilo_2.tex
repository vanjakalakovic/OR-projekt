\documentclass[a4paper,12pt]{article}
\usepackage[slovene]{babel}
\usepackage[utf8]{inputenc}
\usepackage[T1]{fontenc}
\usepackage{lmodern}
\usepackage{amsmath}
\usepackage{amsfonts}
\usepackage{amssymb}
\usepackage{enumitem}
\usepackage{fancyhdr}
\usepackage{hyperref}  
\usepackage{amsthm} 
\begin{document}
\thispagestyle{empty}
{\large
\noindent Univerza v Ljubljani\\[1mm]
Fakulteta za matematiko in fiziko\\[5mm]}
\vfill

\begin{center}{\Large
{\bf Predstavitev 4-ciklov v snarkih}\\[2mm]
Eva Strašek in Vanja Kalaković\\[10mm]
Mentorja: Janoš Vidali, Riste Škrekovski\\[2mm]
Predmet: Finančna matematika \\[2mm]}
\end{center}
\vfill

{\large
Ljubljana, 2023}
\pagebreak

\section{Uvod}

\noindent \textbf{Definicija:} Stopnja vozlišča $v$ v grafu G (oznaka je $deg_G(v)$)
je enaka številu povezav grafa G, ki imajo vozlišče $v$ za svoje krajišče,
pri čemer štejemo zanke dvakrat.\\

\noindent \textbf{Definicija:} Če so vsa vozlišča grafa G enake stopnje
k, pravimo, da je graf \textbf{k-regularen}; 3-regularnim grafom 
pravimo tudi \textbf{kubični grafi}.\\

\noindent \textbf{Definicija:} Graf je \textbf{povezan}, če za poljubni vozlišči obstaja
pot med njima, sicer je graf \textbf{nepovezan}.\\

\noindent \textbf{Definicija: Komponenta} grafa G je maksimalen povezan podgraf
grafa G.\\

\noindent \textbf{Definicija:} Vozlišče $v \in V(G)$ je \textbf{prerezano vozlišče}, če ima 
podgraf G - $v$ več komponent kot graf G.\\

\noindent \textbf{Definicija:} Povezavi, ki ima za krajišči prerezani vozlišči,
pravimo \textbf{prerezana povezava} ali \textbf{most}.\\

\noindent \textbf{Definicija: Kromatično število} grafa G (oznaka $\chi$(G)) je najmanjše število barv, ki jihpotrebujemo, da vozlišča grafa pobarvamo tako, da so vsa sosednja
vozlišča pobarvana paroma različno.\\

\noindent \textbf{Definicija: Kromatični indeks} grafa G je število $\chi'$(G), ki predstavlja najmanjše
število barv, ki jih potrebujemo za barvanje povezav grafa G tako, da so sosednje povezave pobarvane paroma različno.\\

\noindent \textbf{Vizingov izrek:} Za enostaven graf G z maksimalno stopnjo $\Delta$(G) velja
$\Delta$(G) $\leq \chi$'(G) $\leq \Delta$(G) + 1. \\

\noindent \textbf{Definicija:} Če so vsa vozlišča $v_0,v_1, \dots, v_k$ različna,
govroimo o poti, v primeru, ko pa so vsa vozlišča različna, razen $v_0 = v_k$,
imamo opravka z \textbf{obhodom}. Dolžini najkrajšega obhoda v grafu G pravimo
\textbf{notranji obseg}.\\

\noindent \textbf{Definicija:} Naj bo A množica prerezanih povezav moči 3 in G graf.\\
 Če G - A predstavlja dve komponenti, ki vsebujeta cikel, pravimo da je A ciklični prerez.\\

\noindent \textbf{Definicija: Ciklična povezavna povezanost} grafa G (zapis $\lambda_c$(G)) je velikost najmanjšega cikličnega prereza grafa G. Pravimo, da je G ciklično 
k-povezavno povezan, če je $\lambda_c$(G) $\geq$ k. Ali drugače, grafu G moramo odstraniti najmanj k-povezav, da nam ta razpade na dve komponenti, ki vsebujeta cikel.\\

\noindent \textbf{Definicija:} Snark je ciklično 4-povezavno povezan kubičen graf
z notranjim obsegom vsaj 5 in kromatičnem indeksom 4. \\

\newpage

\section{Načrt dela}

\noindent \subsection*{\large Naloga:} Želimo preveriti, ali (in kdaj) uvedba 4-ciklov v snark ohranja kromatični
indeks(tj. kromatični indeks ostane 4). Uvedbo 4-cikla lahko izvedemo
vsaj na dva načina:
\begin{enumerate}
    \item Vzemite dva robova ab in cd v G in ju dvakrat razdelite, tako da dobite
    pot au1u2b iz roba ab in pot cv1v2d iz roba cd. Nato povežite u1 z v1
    in u2 z v2.
    \item Naj bo ab rob v G in a1, a2 druga dva soseda a, in naj bosta b1, b2
    druga dva soseda b. Zdaj odstranimo vrhova a, b in povežemo a1 z b1
    in a2 z b2.
\end{enumerate}

\noindent \subsection*{Načrt:} Najprej bova prenesli majše snarke iz House Of Graphs na katerih bova preverjali ohranjanje kromatičnega indeksa.
Nato bova napisali program, ki bo na prenesenih snarkih uvedel 4-cikle na zgoraj opisana načina.\\
Novo nastale snarke bova potem shranili in s programom izračunali nove kromatične indekse, kjer bova izločili tiste ki so različni od 4.\\
Na ta način bova ugotovili ali uvedba 4-cikla v snark ohrani kromatični indeks in v katerih primerih to drži. Pri tem si bova pomagali z naslednjo lemo: \\

\noindent \textbf{Lema:} Naj bo graf G' dobljen tako, da kubičnemu grafu G dodamo vozlišča $a_1,a_2$ in $a_3,a_4$ zaporedno na povezavi $e_1 = u_1v_1$ in $e_2 = u_2v_2$
ter povezave $a_1a_4$ in $a_2a_3$. Če je $\chi'$(G') = 4, potem je $\chi'$(G) = 4.
\end{document}